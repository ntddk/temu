\documentclass[11pt,onecolumn]{article}
\usepackage{fancyvrb}
\usepackage{fullpage}
\usepackage{amsmath} % for \text; HeVeA doesn't know about amstext
\usepackage{hevea}
\usepackage{comment}
\usepackage[dvips]{graphicx}       %%% graphics for dvips
\usepackage[colorlinks=true, linkcolor=blue, 
  citecolor=blue, urlcolor=blue,
  ps2pdf,                %%% hyper-references for ps2pdf
  bookmarks=true,        %%% generate bookmarks ...
  bookmarksnumbered=true,%%% ... with numbers
]{hyperref}
% pdfcreator and pdfproducer are set automatically in pdfLaTeX
\hypersetup{ pdfcreator  = {LaTeX with hyperref package},
  pdfproducer = {dvips + ps2pdf} }
\begin{latexonly}
% HeVeA can't handle this:
\let\url\nolinkurl % because dvips cannot break url across lines
\end{latexonly}
%\usepackage{times}

%\includecomment{svn}
%\excludecomment{release}
\includecomment{release}
\excludecomment{svn}

\begin{svn}
\newcommand{\sreither}[2]{#1}
\end{svn}
\begin{release}
\newcommand{\sreither}[2]{#2}
\end{release}

% Make \_ be CMTT's underscore, not \textunderscore, since we use
% it in code and command names.
\def\_{\char"5F}
\newcommand{\mytt}{\small \tt}
\newcommand{\titled}{TEMU installation and user manual}
\title{\mbox{}\\[-.8in]\bf \titled}
\author{BitBlaze Team}
\date{Nov 5th, 2009:
\sreither{SVN trunk r1727}{Release 1.0}
and Ubuntu 9.04}
\begin{document}
\maketitle

\tableofcontents

\section{Introduction}

This document is a quick start guide for setting up and running TEMU,
the dynamic tracing component of the BitBlaze Binary Analysis
Framework. It assumes that you have some familiarity with Linux.  The
instructions are based on
\begin{svn}
the version of TEMU in the SVN trunk as of
the date shown in the header,
\end{svn}
\begin{release}
the release of TEMU shown in the header,
\end{release}
running on a vanilla Ubuntu 9.04
distribution of Linux. We intermix instructions with explanations
about utilities to give an overview of how things work. The goal in
this exercise is to take a simple program, trace it on some input and
treat its keyboard input as symbolic. You can then use the generated
trace file in the separate Vine tutorial.

\section {\label{sec:install}Installation}

The following script shows the steps for
building and installing TEMU and the other software it depends on:
\begin{svn}
(This is also found as
\verb'docs/install-temu-svn.sh' in the TEMU source, 

\VerbatimInput{../install-temu-svn.sh}
\end{svn}
\begin{release}
(This is also found as
\verb'docs/install-temu-release.sh' in the TEMU source, 

\VerbatimInput{../install-temu-release.sh}
\end{release}

\section {\label{sec:vmconfig}Configuring a new VM}
While QEMU itself is compatible with almost any guest OS that runs on
x86 hardware, TEMU requires more knowledge about the OS to bridge the
semantic gap and provide information about OS abstractions like
processes.
For Linux, we embed knowledge about kernel data structures directly
into TEMU; the same approach could potentially be used for Windows,
but TEMU's current Windows support uses an extra driver that runs
within the guest.
This release of TEMU works out-of-the-box with VMs running Ubuntu
Linux 9.04 32-bit.
A few extra steps are required to support Windows XP or other versions
of Linux.

\begin{itemize}
\item \textbf{Windows-based VMs}
TEMU supports Windows XP (we've tested with SP1, SP2, and SP3), with
the installation of a support driver.
(We have not tested versions prior to XP, and Windows Vista or Windows
7 are not supported.)
The driver is found in both source and binary form in the
\texttt{testdrv/driver} directory of the TEMU release.
To install the driver, first copy the \texttt{testdrv.sys} driver file
into the \verb'%SYSTEM32%\drivers' directory (i.e., typically 
\verb'C:\Windows\system32\drivers').
Then, double-click the \texttt{testdrv.reg} file to copy its contents
into the registry to configure the driver; it will then be loaded on
the next reboot.
To confirm that the driver is working correctly, look for a
\texttt{guest.log} file created in the directory where you are running
TEMU; it shows some of the data collected by TEMU.

\item \textbf{Linux-based VMs}
Because TEMU's Linux support requires more intimate knowledge of OS
internals, it is more version-dependent than Windows support.
TEMU's \texttt{kernel\_table} data structure, found in
\texttt{shared/read\_linux.c} in the source, contains information
about the location and layout of kernel global data, and the addresses
of functions whose execution to monitor; unfortunately this
information is different for different kernel versions and Linux
distributions.
As distributed, TEMU supports the kernel from a recent version of
Ubuntu Linux 9.04, as well as some older ones, but you must collect
the information anew to support a new kernel or distribution version.

Most of this information (all, for some 2.4 kernels) can be collected
automatically using a kernel module whose source is found in the
\texttt{shared/kernelinfo} directory.
There are several sample variants for different distribution versions;
\texttt{procinfo-ubuntu-hardy}, which was originally created for Ubuntu
8.04 and also works for 9.04, would be a good starting point for
modern 2.6-based systems.
Copy the module source to your guest VM, and compile it there (you
should have the kernel header files matching the running kernel
installed).
Then, load the module using the \texttt{insmod} command, and look for
its output in the kernel's logs (e.g., \texttt{/var/log/kern.log}) or
the kernel log ring buffer (displayed by the \texttt{dmesg} command).
Then copy these entries to \texttt{shared/read\_linux.c} and recompile
TEMU.
For 2.6 kernels, we haven't been able to find an appropriate hooking
function that is exported to modules, so you'll need to find the
address of a function that is called after a new process is created
using the kernel's symbol table (usually kept in a file like
\texttt{/boot/System.map-2.6.28-15-generic}), and add it as the second
value in the information structure by hand.
For recent kernels, we've found the function
\texttt{flush\_signal\_handlers} works well.
\end{itemize}

After performing the above steps, you can check that things are OK by 
running the \texttt{guest\_ps} (Windows) or \texttt{linux\_ps} (Linux) command, and 
verifying that the current processes are correctly displayed; an error in the configuration 
will likely cause this command to output garbage, or cause TEMU to crash/hang.

\section {Setting up TEMU network}
\label {sec:setup}

Running QEMU by itself should be the first step, before you try to run
TEMU. There are many platform specific tweaks that you may need in
order to get QEMU usable for your project. Though not needed for this
excercise, you will often need to set up a network inside the QEMU
image that you use. You may skip this network setup section, if you
will not need this.

This document does not intend to go into great depth in setting up
QEMU itself.  But we describe some mechanisms that have worked for
us. You may need a bit Googling to set this up on your specific
platform and network configuration.

\begin{itemize}

\item \textbf{Method 1} - User-level network emulation

The simplest kind of network emulation, which QEMU performs by
default, uses just user-level network primitives on the host side, and
simulates a private network for the virtual machine. This is
sufficient for many utility purposes, such as transferring files to
and from the virtual machine, but it may not be accurate enough for
some kinds of malicious network use. The QEMU options for enabling
this mode explicitly are
\verb'-net nic -net user,hostname=mybox', where \verb'mybox' is the
hostname for the virtual DHCP server to provide to the VM.

If you want to connect to well-known services on the VM, you'll need
to redirect them to alternate ports on the host with the
\verb'-redir' option. For instance, to make it possible to SSH to a
server on the VM, give QEMU the option \verb'-redir tcp:2022::22',
then tell your SSH client to connect to port 2022 on the local
machine.

\item \textbf{Method 2} -  Use tap network interface.
\begin{Verbatim}[frame=lines, framesep=.5em]
Create a script /etc/qemu-ifup, including the following lines. Be sure to make 
this script executable.
#!/bin/sh
sudo /sbin/ifconfig $1 192.168.10.1

You must then setup a tap interface. This step can be skipped if you
are willing to run QEMU as root.
$ sudo apt-get install uml-utilities
$ sudo /usr/sbin/tunctl -b user -t tap0

Start the Windows VM. The host machine will have the IP address
192.168.10.1, as is specified in the above script.
$ sudo chmod 666 /dev/net/tun
$ qemu -kernel-kqemu -snapshot -net nic,vlan=0 \
  -net tap,vlan=0,script=/etc/qemu-ifup \
  -monitor stdio /path/to/qemu/image
  
If you don't want to type these commands each time you start TEMU,
you can create a wrapper script which initializes the network,
starts TEMU with desired command-line arguments, then removes the
tap interface once TEMU exits.
\end{Verbatim}
%$

\end{itemize}

Once QEMU is set up and running, TEMU should run in the same way. You
can run TEMU's \texttt{qemu} as root, just the same way as you run
QEMU using the installed \texttt{qemu} in the PREFIX directory.




\section {Taking traces}

Assuming that you have compiled TEMU and you have identified the
command line to launch your QEMU session, we can now go ahead and try
out a simple example trace. Here we demonstrate the procedure for a
Ubuntu 9.04 Linux image; the commands are mostly the same for a Windows
image.

The command-line options for TEMU are mostly the same as for
QEMU. Besides whatever options are needed for your virtual machine to
run correctly, the example below adds two more. \verb'-snapshot' tells
QEMU not to write changes to the virtual hard disk back to the disk
image file unless explicitly requested, so you don't have to worry
about messing up your VM with experiments gone awry.
\verb'-monitor stdio' tells QEMU to put up a command-line prompt on
your terminal, which we will use to give TEMU commands. 

A  command line to launch TEMU looks like:

\begin{Verbatim}[frame=lines, framesep=.5em]
% cd ~/bitblaze/temu
% ./tracecap/temu  -snapshot -monitor stdio ~/images/ubuntu904.qcow2
\end{Verbatim}

The output on the console is:

\begin{Verbatim}[frame=lines, framesep=.5em]
QEMU 0.9.1 monitor - type 'help' for more information
(qemu)
\end{Verbatim}

You may also see a warning indicating that \verb'kqemu' is disabled
for one reason or another; these may mean that your VM will run more
slowly, but can otherwise be ignored.

\begin {enumerate}
  \item \emph {Generate a simple program in the QEMU image:}
    In the guest Linux session, create a \texttt{foo.c} program as
    follows, and start it:
\begin{Verbatim}[frame=lines, framesep=.5em]
$ cat foo.c
#include <stdio.h>

int main(int argc, char **argv)
{
  int x;
  scanf("%d", &x);
  if (x != 5)
      printf("Hello\n");
  return 0;
}
$ gcc foo.c -o foo
$ ./foo

\end{Verbatim}
%$
\item \emph {Load the TEMU plugin}

At the \verb'(qemu)' prompt, say:
\begin{Verbatim}[frame=lines, framesep=.5em]
(qemu) load_plugin tracecap/tracecap.so
Cannot determine file system type
tracecap/tracecap.so is loaded successfully!
(qemu) enable_emulation
Emulation is now enabled
\end{Verbatim}

The warning about \verb'Cannot determine file system type' applies to
functionality we won't be using, and can be
ignored. \verb'enable_emulation' is required to activate any of TEMU's
per-instruction tracing hooks; without it, later steps won't see any
of the instructions executed.

\item \emph {Find out the process id you the program you want to trace:}

In the \verb'(qemu)' prompt, run the \texttt{linux\_ps} command to
find the process id of the \texttt{./foo} application running in the
guest Linux image.

\begin{Verbatim}[frame=lines, framesep=.5em]
(qemu) linux_ps
    0  CR3=0x00000000  swapper
    1  CR3=0xC7DEA000  init
         0x08048000 -- 0x0804E000 init
         0x0804E000 -- 0x0804F000 init
         0x0804F000 -- 0x08053000 
         0x40000000 -- 0x40013000 ld-2.2.5.so
         0x40013000 -- 0x40014000 ld-2.2.5.so
         0x40022000 -- 0x40023000 
         0x42000000 -- 0x4212C000 libc-2.2.5.so
         0x4212C000 -- 0x42131000 libc-2.2.5.so
         0x42131000 -- 0x42135000 
         0xBFFFD000 -- 0xC0000000 
	 .....
  958  CR3=0xC51A1000  foo
         0x08048000 -- 0x08049000 foo
         0x08049000 -- 0x0804A000 foo
         0x40000000 -- 0x40013000 ld-2.2.5.so
         0x40013000 -- 0x40014000 ld-2.2.5.so
         0x40014000 -- 0x40015000 
         0x42000000 -- 0x4212C000 libc-2.2.5.so
         0x4212C000 -- 0x42131000 libc-2.2.5.so
         0x42131000 -- 0x42135000 
         0xBFFFE000 -- 0xC0000000 
	 ....
\end{Verbatim}
  
The PID, here 958, is shown on the header line for the named process.
The other information isn't relevant for what we're doing, but if
you're curious, the \verb'CR3' value is a pointer to the kernel-space
page table for each process, and the remaining lines show the virtual
address ranges for the process's various segments (mappings), which
are either text or data segments from executables or shared libraries,
or anonymous heap or stack areas.

For a Windows image, you need to run the \texttt{guest\_ps} command
instead of \texttt{linux\_ps}.

\item \emph {Trace the process, and record the instructions it executes in a file:}

The  \texttt{trace} command takes  the process  id and  the name  of a
trace file to write information into, as shown below.

\begin{Verbatim}[frame=lines, framesep=.5em]
(qemu) trace 958 "/tmp/foo.trace"
PID: 958 CR3: 0x06301000
PROTOS_IGNOREDNS: 0, TABLE_LOOKUP: 1 TAINTED_ONLY: 0
 TRACING_KERNEL_ALL: 0 TRACING_KERNEL_TAINTED: 0 TRACING_KERNEL_PARTIAL: 0
\end{Verbatim}

As an alternative to steps 3-4 above, you can also tell TEMU to begin 
tracing a program before you've loaded it, with the \texttt{tracebyname} 
command. This command will monitor new processes and automatically trace 
the next instance of the target program. Example usage of this command is 
shown below.

\begin{Verbatim}[frame=lines, framesep=.5em]
(qemu) tracebyname foo "/tmp/foo.trace"
Waiting for process foo to start
$ ./foo
(qemu) PID: 472 CR3: 0x0a025000
Tracing foo
\end{Verbatim}

\item \emph {Specify what input to taint, and give the input:}

With the \texttt {taint\_sendkey} command we can send input to the
traced process, and also mark this input as tainted.  The taint
tracking engine will perform dynamic taint tracking, i.e. mark all
data derived from tainted input as tainted.  If any of the operands of
an executed instruction are tainted, the result is also marked
tainted.  This command takes 2 arguments -- the character (really,
keyboard key) to give as input (\texttt{5} in the example below) and
an identifier to identify this input in the trace (given by ``1001''
in the trace; it should not be zero).
The trace of this process will log all data read and
written at each instruction, the instruction itself, and the
associated data taint in the trace file.

\begin{Verbatim}[frame=lines, framesep=.5em]
(qemu) taint_sendkey 5 1001
Tainting keystroke: 9 00000001
(qemu) taint_sendkey ret 1001
Tainting keystroke: 9 00000001
Time of first tainted data: 1197072993.761231
(qemu) 
\end{Verbatim}

Note that TEMU is tracking taint throughout the whole simulated
machine, but only tracing in the requested process. The \texttt{first
tainted data} message refers to the traced program, and doesn't show
up until a complete line has been typed, because the operating system
is buffering the input line before that.

\item \emph {Stop tracing and tainting:}

We  are  done with  tainting  and tracing,  so  we  use the  following
commands to turn off the components.

\begin{Verbatim}[frame=lines, framesep=.5em]
(qemu) trace_stop
Stop tracing process 958
Number of instructions decoded: 5979
Number of operands decoded: 13349
Number of instructions written to trace: 5890
Number of tainted instructions written to trace: 85
Processing time: 0.464029 U: 0.444028 S: 0.020001
Generating file: /tmp/foo.trace.functions
(qemu) unload_plugin
Emulation is now disabled
protos/protos.so is unloaded!
\end{Verbatim}

\end {enumerate}

At the end, you should have a trace file generated at the file name
you specified (\verb'/tmp/foo.trace' in the example). The trace has a
specific binary format which is not human-readable, but you can check
that it contains some data (it should be between about 100k and 800k
for this example).  It
contains instructions, concrete values of the operands seen in the
execution of the program, and the associated taint value.

As an aside, if you want to generate traces with network input rather
than keystrokes you can follow the same steps but with two
changes. First, after the plugin is loaded, issue the command
\verb'taint_nic 1' to tell TEMU to mark all input
received from the network card to as tainted. Second, instead of
giving \texttt{taint\_sendkey}, just simply direct the input to the IP
address/port of the virtual machine. If the input causes the EIP to
become tainted, TEMU will immediately write all trace data and
quit. You can use this to launch network attacks on programs in the
guest OS image.









\section {Troubleshooting}
This section describes some problems users have experienced when using TEMU, 
along with the most common causes of these problems.

\begin {enumerate}
  \item \emph {TEMU does not begin tracing program}
    \begin{itemize}
    \item {Did you remember to \texttt{enable\_emulation} before running the program?} 
    \item {Did you enter to correct PID (\texttt{trace} command) or correct filename 
    (\texttt{trace\_by\_name} command)?}     
    \item {Are you using a supported operating system? TEMU has been preconfigured for
    Ubuntu Linux 9.04, and requires a driver to be installed on Windows systems.}     
    \end{itemize}
  \item \emph {Generated trace file is empty}
    \begin{itemize}
    \item {Did you remember to run \texttt{trace\_stop}?}     
    \item {Are you using the \emph{tracecap} plugin? The \emph{tracecap} can be configured 
    to write only certain types of instructions (for instance, tainted instructions) to the trace file. 
    Check the plugin settings in the main.ini file.}       
    \item {Did you load any HOOK files? Configuration settings in HOOKs may sometimes 
    disable writing to the trace file until certain trigger conditions are met.} 
    \end{itemize}  
  \item \emph {No tainted instructions were written to the trace file}
    \begin{itemize}
    \item {Was any tainted data accessed by the traced program?}
    \item {Are you loading tainted data from hard drive? Caching by the OS sometimes 
      causes data from primary hard disk to be ``missed'' by TEMU. Try loading the tainted data 
      from a secondary hard disk.} 
    \end{itemize}   
  \item \emph {Compile warnings about {\tt fastcall}}
    \begin{itemize}
    \item {Are you trying to compile with GCC 3.3? It isn't supported.}
    \end{itemize}
  \item \emph {Missing symbols starting with {\tt \_sch\_}}
    \begin{itemize}
    \item {These indicate a problem linking with the GNU
        Binutils. Make sure you have matching development and runtime
        versions of its libraries installed, and that {\tt
          /usr/lib/libbfd.so} exists.}
    \end{itemize}
  \item \emph {TEMU can't find a BIOS image or keymap}
    \begin{itemize}
    \item {Either run {\tt make install} to put these in the locations
        TEMU is expecting, or give their locations with the {-L} flag.}
    \end{itemize}
  \item \emph {{\tt linux\_ps} loops or prints garbage}
    \begin{itemize}
    \item {This can be caused by TEMU having incorrect information
        about your Linux kernel. Check that the version you are
        running is one of the already supported ones, or provide that
        information as described in Section~\ref{sec:vmconfig}}
    \end{itemize}
\end {enumerate}

\section {Acknowledgements}

TEMU's Tracecap plugin links with OpenSSL (copyright 1998-2004 the
OpenSSL Project), Sleuthkit (portions copyright 1997-1999 IBM and
other authors), XED (copyright 2004-2009 Intel), and llconf (copyright
2004-2007 Oliver Kurth). However like TEMU itself our redistribution
of that code is WITHOUT ANY WARRANTY.

\section {Reporting Bugs}

\begin{svn}
Please report bugs to the bugzilla at:
\texttt{https://bullseye.cs.berkeley.edu/bugzilla/}.

When reporting bugs in the SVN version of the software, it's most
useful if you can reproduce your problem with the most recent trunk
version, but if you're using an older version, please specify the
revision number (i.e., the output of the svnversion command in your
bug report). And please also report if you notice something wrong or
out of date in this document.
\end{svn}
\begin{release}
Though we cannot give any guarantee of support for TEMU, we are
interested in hearing what you are using it for, and if you encounter
any bugs or unclear points.
%
Please send your questions, feature suggestions, bugs (and, if you
have them, patches) to the bitblaze-users mailing list.
%
Its web page is:
\url{http://groups.google.com/group/bitblaze-users}.
\end{release}

\end{document}
